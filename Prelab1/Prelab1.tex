\documentclass[a4paper, 11pt]{article}
\title{FYS2150 Prelab 1}
\author{Gunnar Lange \\ Brukernavn: gunnalan}
\usepackage{amsmath}
\usepackage{amssymb}
\usepackage{amsfonts}
\usepackage{graphicx}
\usepackage{tikz}
\usetikzlibrary{shapes.geometric, arrows}
\date{}
\usepackage[T1]{fontenc}
\usepackage[utf8]{inputenc}
\usepackage{lmodern}
\usepackage[english]{babel}
\usepackage{listings}
\usepackage{color} %red, green, blue, yellow, cyan, magenta, black, white
\definecolor{mygreen}{RGB}{28,172,0} % color values Red, Green, Blue
\definecolor{mylilas}{RGB}{170,55,241}	

\lstset{language=Matlab,%
    %basicstyle=\color{red},
    breaklines=true,%
    morekeywords={matlab2tikz},
    keywordstyle=\color{blue},%
    morekeywords=[2]{1}, keywordstyle=[2]{\color{black}},
    identifierstyle=\color{black},%
    stringstyle=\color{mylilas},
    commentstyle=\color{mygreen},%
    showstringspaces=false,%without this there will be a symbol in the places where there is a space
    numbers=left,%
    numberstyle={\tiny \color{black}},% size of the numbers
    numbersep=9pt, % this defines how far the numbers are from the text
    emph=[1]{for,end,break},emphstyle=[1]\color{red}, %some words to emphasise
    %emph=[2]{word1,word2}, emphstyle=[2]{style},    
}

\usepackage[most]{tcolorbox}

\tcbset{
    frame code={}
    center title,
    left=0pt,
    right=0pt,
    top=0pt,
    bottom=0pt,
    colback=gray!20,
    colframe=white,
    width=\dimexpr\textwidth\relax,
    enlarge left by=0mm,
    boxsep=5pt,
    arc=0pt,outer arc=0pt,
    }
\begin{document}

\maketitle
\begin{enumerate}
\item \textbf{B}
\item \textbf{B}
\item \textbf{B}
\item \textbf{A}, \textbf{C}, \textbf{D}
\item \textbf{A}
\item \textbf{A}
\item \textbf{A}
\item \textbf{C}
\item \textbf{A}
\item \textbf{A}
\item \textbf{A}
\item \textbf{B}
\item \textbf{A}
\item \textbf{A}
\item \textbf{A}
\item \textbf{A}
\newpage
\item Skriptet mitt er vist nedenfor:
\begin{lstlisting}
A=5; %Svingeamplitude
omega=50; %Svingefrekvens
frequency=1000; %Samplingsfrekvens
n=2000; %Svarer til 2sek
sigma=0.2*A; %For Gauss noise
mu=0;
linear_trend_factor=0.1*A; %Storrelse paa drift
x=linspace(0,1.0*frequency/n, n);
noise=normrnd(mu,sigma, [1 n]);
linear_drift=linspace(0, linear_trend_factor, n);
signal=A*sin(omega*x);
Data=signal+noise+linear_drift;

%Plot av data
figure(1)
plot(x,Data);
title('Maalinger');
xlabel('Sekunder, s [s]');
ylabel('Volt, V [V]');

%Histogram
bins=n/20.0;
figure(2)
histogram(Data, bins); %Histogram av alle data
title('Histogram av maalingene');
xlabel('Volt, V [V]');
ylabel('Absolutt frekvens');

%%%%Kommentar til fordeling%%%
%Ser at fordelingen er bimodal, med topper ved maksimum av kurven%
%Her er den deriverte minst, og siden samplingsfrekvensen er konstant,
%endrer sysmtemet seg minst her. Dermed er det flest datapunkter her. Merk
%ogsaa at gjennomsnittet ikke ser ut til aa vaere noyaktig null, baade pga.
%noise og pga. drift
%%%

%Histogram av stoy
figure(3)
gaussian_noise=Data-signal-linear_drift;
title('Histogram av stoyet');
xlabel('Volt, V [V]');
ylabel('Absolutt frekvens');
histogram(gaussian_noise) %Histogram kun av stoy

%Normalisering av stoy histgoramet
noise_data=histogram(Data-signal-linear_drift, bins);
noise_values=noise_data.Values;
noise_bins=noise_data.BinEdges(1:end-1)+noise_data.BinWidth/2;
noise_bins=sigma*noise_bins+mu; %Normaliser variablene
noise_values=sigma*noise_values+mu;
area=sum(noise_values.*(noise_bins(2)-noise_bins(1))); %Verdiene*bredde

figure(4)
scatter(noise_bins,noise_values/area);
hold on
plot(xhist, normpdf(xhist, 0, 1)) %Sammeling med standard normalfordeling
title('Normalisert histogram av stoyet, med normalfordeling');
xlabel('Volt, V [V]');
ylabel('Relativ frekvens');
legend('Data', 'Normal distribution');


%Ta ut lineaer trend - bruker innebyd Matlab funksjon
figure(5)
trend=detrend(Data);
hold off
plot(x, trend);
title('Maalinger');
xlabel('Sekunder, s [s]');
ylabel('Volt, V [V]');
\end{lstlisting}

\end{enumerate}

\end{document}